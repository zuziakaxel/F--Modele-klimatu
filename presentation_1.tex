\documentclass{beamer}

\mode<presentation> {




%\usetheme{Antibes} %ok
%\usetheme{Berlin} %ok
%\usetheme{Boadilla} %ok
\usetheme{Darmstadt} %ok
%\usetheme{Dresden} %ok
%\usetheme{Frankfurt} %ok
%\usetheme{Ilmenau} %ok
%\usetheme{Pittsburgh} %ok
%\usetheme{Rochester} %ok


%\usecolortheme{dolphin}
%\usecolortheme{orchid}
%\usecolortheme{whale}



}


\usepackage[utf8]{inputenc}
\usepackage[OT4]{polski}
\usepackage{tabularx}

\usepackage{graphicx} % Allows including images
\usepackage{booktabs} % Allows the use of \toprule, \midrule and \bottomrule in tables



\title[Modelowanie klimatu]{Modelowanie klimatu} % The short title appears at the bottom of every slide, the full title is only on the title page

\author{Axel Zuziak, Marcin Węglarz} % Your name
\institute[AGH WFiIS]
{
AGH WFiIS \\
Fizyka Techniczna\\ % Your institution for the title page
\medskip
}
\date{\today} % Date, can be changed to a custom date

\begin{document}
%progress bar in footline *************************************************************************

\definecolor{lightgr}{rgb}{0 0.4 0.9}
\makeatletter
\addtobeamertemplate{footline}{%
	\color{lightgr}% to color the progressbar
	\hspace*{-\beamer@leftmargin}%
	%\rule{\beamer@leftmargin}{0pt}%
	\rlap{\rule{\dimexpr
			\beamer@startpageofframe\dimexpr
			\beamer@rightmargin+\textwidth\relax/\beamer@endpageofdocument}{3pt}} %grubosc paska
	% next 'empty' line is mandatory!
	
	\vspace{0\baselineskip}
	{}
} %koniec progress bara **************************************************************************************



%++++++++++++++++++++++++++++++++++++++++++++++++++++++++++++++++++++++++
%notatki
%str 47-63
%str 117-150
%str 165-200
%str 244-246

%++++++++++++++++++++
%s.176
%TODO
%Fizyka + równania
%Implementacja tych równań 
%Typy modeli (0,1,2,3 wymiarowe + kompletne)
%s.60





















\begin{frame}
\titlepage % Print the title page as the first slide
\end{frame}

%\begin{frame}
%\frametitle{Overview} % Table of contents slide, comment this block out to remove it
%\tableofcontents % Throughout your presentation, if you choose to use \section{} and \subsection{} commands, these will automatically be printed on this slide as an overview of your presentation
%\end{frame}

%----------------------------------------------------------------------------------------
%	PRESENTATION SLIDES
%----------------------------------------------------------------------------------------

%------------------------------------------------
%\section{First Section} % Sections can be created in order to organize your presentation into discrete blocks, all sections and subsections are automatically printed in the table of contents as an overview of the talk
%------------------------------------------------

%\subsection{Subsection Example} % A subsection can be created just before a set of slides with a common theme to further break down your presentation into chunks

\begin{frame}
\frametitle{Co to jest klimat?}
Klimatem nazywamy średnie warunki pogodowe obserwowane w danym miejscu na przestrzeni lat. Przykładowe czynniki: temperatura, opady, zachmurzenie, wilgotność.
Modele klimatu są uproszonym opisem skomplikowanych procesów.
Klimat dzielimy na pięć części:
\begin{itemize}
	\item \textbf{Atmosfera} Gazowa część ponad powierzchnią ziemi.
	\item \textbf{Hydrosfera} Wszystkie formy wody nad i pod powierzchnią ziemi.
	\item \textbf{Kriosfera} Wszystkie formy wody w postaci lodu.
	\item \textbf{Powierzchnia lądowa}
	\item \textbf{Biosfera} Organizmy żyjące w hydrosferze oraz na powierzchni lądowej.
\end{itemize}
\end{frame}


\begin{frame}
	\frametitle{}
	\begin{figure}[h]
		\begin{center}
			\includegraphics[width=0.7\linewidth]{images/Figure1}
			\caption{Czynniki definiujące i wpływające na klimat}
		\end{center}
	\end{figure}
\end{frame}

\begin{frame}
	\frametitle{Zerowymiarowy model cieplarniany}
	\begin{figure}[h]
		\begin{center}
			\includegraphics[width=0.7\linewidth]{images/0D_Model.png}
			\caption{Zerowymiarowy model bilansu promieniowania}
		\end{center}
	\end{figure}

\end{frame}



\begin{frame}
	\frametitle{Matematyczne spojrzenie na bilans energetyczny}
	\begin{block}{Bardzo prosty model bilansu radiacyjnego}
		\[(1-a)\frac{S}{4} = \sigma T_a^4 + t\sigma T_s^4
		\]
	\end{block}
	\begin{block}{Bilans dla powierzchni Ziemi}
		\[(-t_a)(1-a_s)\frac{S}{4}+c(T_s - T_a)+\sigma T_a^4 =0
		\]
	\end{block}
	\begin{block}{Bilans dla atmosfery}
		\[-(1- a_a-t_a+a_st_a)\frac{S}{4} - c(T_s - T_a) - \sigma T_s^4
		(1-t_a^{'}-a_a^{'}) + 2\sigma T_a^4=0
		\]
	\end{block}
	%Literką \it{a} oznaczamy albedo, natomiast \it{t} oznacza przepuszczalność. 
	%odpowiednie indeksy oznaczają wartość dla atmosfery(a) lub powierzchni Ziemi(s).
	\scriptsize{(Wartości z primem to wartości dla fal długich.)}
	
\end{frame}

\begin{frame}
	\frametitle{Hipotetyczne przykłady}
	\begin{itemize}
		\item \textit{Współczesna Ziemia.} ($a=0,30, T_s=288K$)
		\item \textit{Biała Ziemia.} ($a = 0,50;T_s=268K$)
		\item \textit{Zima nuklearna.} ($a=0,35; T_s=284K$)
	\end{itemize}
	
\end{frame}

\begin{frame}
	\frametitle{Wymuszenie radiacyjne i sprzężenie zwrotne.}
	\textbf{Wymuszeniem radiacyjnym} nazywamy zjawisko zmiany temperatury na powierzchni Ziemi celem wyrównania bilansu radiacyjnego. 
	\begin{block}{Wzory do ilościowego opisu zmian temperatury}
		\[\Delta I = \frac{\partial I}{\partial T_s}\Delta T_s
		\]
		\[\frac{\partial I}{\partial T_s} = \frac{4}{T_s}(1-a)\frac{S}{4}
		\]
		
		%z tego modelu dI 3,1 W/m(-2)K(-1)
		%z dokladniejszych 4,6 - efekt uwzglednienia sprzezenia zwrotnego
	\end{block}
\end{frame}


\begin{frame}
	\frametitle{Przykłady zjawisk wpływających na globalne ocieplenie}

	\textbf{Zjawiska mogące wzmagać globalne ocieplenie:}
	\begin{enumerate}
		\item Topienie się lodów i śniegów.
		
		\item Zwiększenie ilości pary wodnej w powietrzu. 
		($t_a^{'}\nearrow; a_a^{'}\searrow$).
		
		\item Wzrost zachmurzenia.
		
		\item Wzrost $CO^2$ (mniejsza absorpcja przez oceany,
		szybszy rozkład materii).
		
		\item Szybszy wzrost roślin i zmiana albedo.

	\end{enumerate}

\end{frame}


\begin{frame}
	\frametitle{Przykłady zjawisk wpływających na globalne ocieplenie}
	
	\textbf{Zjawiska mogące osłabiać globalne ocieplenie:}
	
	\begin{enumerate}
		
		\item Wzrost zawartości pary wodnej (średni spadek temperatury
		z wysokością maleje).
		
		\item Wzmożony rozwój alg we wszystkich wodach (użycie $CO^2$ 
		do fotosyntezy).
		
	\end{enumerate}
	
\end{frame}



\begin{frame}
	\frametitle{Opóźnienie czasowe ze względu na obecność oceanów}
	
	\begin{itemize}
		\item Wysoka pojemność cieplna wody.
		
		\item Duża powierzchnia wód na Ziemi.
		
		\item Powolne zmiany temperatury.
	
	\end{itemize}
	
	\begin{block}{Zmiana temperatury w wyniku wymuszenia radiacyjnego}
		\[\Delta T_s(t) = G_f\Delta I
		\]
	\end{block}
	
	\begin{block}{Ta sama zmiana po uwzględnieniu opóźnienia czasowego}
		\[\Delta T_s(t) = G_f\Delta I(1-\exp^{-t/\tau _e})
		\]
	\end{block}
	
	$\tau _e \approx 50-100$ lat
	
\end{frame}






\begin{frame}
	\frametitle{Model 3D - atmosfera v2.0}
	\begin{block}{Prawa zachowania pędu}
		
		\[\frac{D\mathbf{v}}{Dt} = -2 \mathbf{\Omega} \times \mathbf{v} - \rho^{-1}
			\nabla p + \mathbf{g} + \mathbf{F}
		\]
		
	\end{block}
	
	\begin{block}{Prawo zachowania masy}
		\[\frac{D\rho}{Dt} = -\rho\nabla \cdot \mathbf{v} + C - E
		\]
	\end{block}
	
	\begin{block}{Prawo zachowania energii}
		\[\frac{DI}{Dt} = -p\frac{D\rho^{-1}}{Dt} + Q
		\]
	\end{block}
	
	\begin{block}{Równanie stanu gazu doskonałego}		
		\[p=\rho RT
		\]
	\end{block}
		

\end{frame}


\begin{frame}
	\frametitle{Implementacja}
	W celu implementacji naszych równań musimy im nadać wartości dyskretne.
	Modelujemy atmosferę, dzieląc ją na pudła.
	
	\begin{figure}[h]
		\begin{center}
			\includegraphics[width=0.7\linewidth]{images/box.png}
			\caption{Model podziału atmosfery na pudła.}
		\end{center}
	\end{figure}
	
\end{frame}


















\begin{frame}
\frametitle{Multiple Columns}
\begin{columns}[c] % The "c" option specifies centered vertical alignment while the "t" option is used for top vertical alignment

\column{.45\textwidth} % Left column and width
\textbf{Heading}
\begin{enumerate}
\item Statement
\item Explanation
\item Example
\end{enumerate}

\column{.5\textwidth} % Right column and width
Lorem ipsum dolor sit amet, consectetur adipiscing elit. Integer lectus nisl, ultricies in feugiat rutrum, porttitor sit amet augue. Aliquam ut tortor mauris. Sed volutpat ante purus, quis accumsan dolor.

\end{columns}
\end{frame}


\begin{frame}
\frametitle{Theorem}
\begin{theorem}[Mass--energy equivalence]
$E = mc^2$
\end{theorem}
\end{frame}


\begin{frame}
\frametitle{References}
\footnotesize{
\begin{thebibliography}{99} % Beamer does not support BibTeX so references must be inserted manually as below
\bibitem[Smith, 2012]{p1} John Smith (2012)
\newblock Title of the publication
\newblock \emph{Journal Name} 12(3), 45 -- 678.
\end{thebibliography}
}
\end{frame}


\begin{frame}
\Huge{\centerline{The End}}
\end{frame}


\end{document} 